%% Generated by Sphinx.
\def\sphinxdocclass{report}
\documentclass[letterpaper,10pt,bahasai]{sphinxmanual}
\ifdefined\pdfpxdimen
   \let\sphinxpxdimen\pdfpxdimen\else\newdimen\sphinxpxdimen
\fi \sphinxpxdimen=.75bp\relax

\PassOptionsToPackage{warn}{textcomp}
\usepackage[utf8]{inputenc}
\ifdefined\DeclareUnicodeCharacter
% support both utf8 and utf8x syntaxes
  \ifdefined\DeclareUnicodeCharacterAsOptional
    \def\sphinxDUC#1{\DeclareUnicodeCharacter{"#1}}
  \else
    \let\sphinxDUC\DeclareUnicodeCharacter
  \fi
  \sphinxDUC{00A0}{\nobreakspace}
  \sphinxDUC{2500}{\sphinxunichar{2500}}
  \sphinxDUC{2502}{\sphinxunichar{2502}}
  \sphinxDUC{2514}{\sphinxunichar{2514}}
  \sphinxDUC{251C}{\sphinxunichar{251C}}
  \sphinxDUC{2572}{\textbackslash}
\fi
\usepackage{cmap}
\usepackage[T1]{fontenc}
\usepackage{amsmath,amssymb,amstext}
\usepackage{babel}



\usepackage{times}
\expandafter\ifx\csname T@LGR\endcsname\relax
\else
% LGR was declared as font encoding
  \substitutefont{LGR}{\rmdefault}{cmr}
  \substitutefont{LGR}{\sfdefault}{cmss}
  \substitutefont{LGR}{\ttdefault}{cmtt}
\fi
\expandafter\ifx\csname T@X2\endcsname\relax
  \expandafter\ifx\csname T@T2A\endcsname\relax
  \else
  % T2A was declared as font encoding
    \substitutefont{T2A}{\rmdefault}{cmr}
    \substitutefont{T2A}{\sfdefault}{cmss}
    \substitutefont{T2A}{\ttdefault}{cmtt}
  \fi
\else
% X2 was declared as font encoding
  \substitutefont{X2}{\rmdefault}{cmr}
  \substitutefont{X2}{\sfdefault}{cmss}
  \substitutefont{X2}{\ttdefault}{cmtt}
\fi


\usepackage[Sonny]{fncychap}
\ChNameVar{\Large\normalfont\sffamily}
\ChTitleVar{\Large\normalfont\sffamily}
\usepackage{sphinx}

\fvset{fontsize=\small}
\usepackage{geometry}


% Include hyperref last.
\usepackage{hyperref}
% Fix anchor placement for figures with captions.
\usepackage{hypcap}% it must be loaded after hyperref.
% Set up styles of URL: it should be placed after hyperref.
\urlstyle{same}
\addto\captionsbahasai{\renewcommand{\contentsname}{Daftar Panduan:}}

\usepackage{sphinxmessages}
\setcounter{tocdepth}{1}



\title{Andieni}
\date{28 Mei, 2020}
\release{1}
\author{Wahid Nurfiantara}
\newcommand{\sphinxlogo}{\vbox{}}
\renewcommand{\releasename}{Rilis}
\makeindex
\begin{document}

\ifdefined\shorthandoff
  \ifnum\catcode`\=\string=\active\shorthandoff{=}\fi
  \ifnum\catcode`\"=\active\shorthandoff{"}\fi
\fi

\pagestyle{empty}
\sphinxmaketitle
\pagestyle{plain}
\sphinxtableofcontents
\pagestyle{normal}
\phantomsection\label{\detokenize{index::doc}}

\begin{quote}

\noindent\sphinxincludegraphics{{andieni_hitam_full}.png}
\end{quote}

Andieni \sphinxhyphen{} \sphinxstylestrong{Aplikasi Naskah Dinas Elektronik Universitas Indonesia}. Merupakan aplikasi yang dibuat dengan tujuan
memudahkan penciptaan naskah dinas (nota dinas dan surat dinas). Memanfaatkan teknologi informasi sejak penyusunan
draft, persetujuan atasan, pengesahan dengan tanda tangan digital, serta distribusi ke tujuan naskah secara elektronik.
Andieni mampu menghadirkan sebuah solusi nyata bagi peningkatan kinerja dan produktifitas UI dalam administrasi persuratan.

Andieni menggunakan teknologi tanda tangan digital atau disebut juga tanda tangan elektronik tersertifikasi (eSign).
Dokumen digital yang telah disematkan sertifikat elektronik akan terenkripsi sehingga segala bentuk manipulasi konten
dokumen akan terlacak dengan menyatakan bahwa dokumen telah diubah sejak ditandatangani terakhir. eSign yang digunakan
pada Andieni merupakan hasil dari kerja sama kemitraan antara UI dengan Balai Sertifikasi Elektronik (BSrE).
Balai Sertifikasi Elektronik merupakan unit pelaksana teknis di Badan Siber dan Sandi Negara (dahulu Lembaga Sandi Negara)
yang memiliki tugas pemberian layanan penerbitan dan pengelolaan sertifikat elektronik di Indonesia untuk instansi pemerintah.

Panduan ini berisi informasi penggunaan andieni bagi pelaksana, sekretaris, dan pimpinan


\chapter{Landasan Kebijakan}
\label{\detokenize{Kebijakan:landasan-kebijakan}}\label{\detokenize{Kebijakan::doc}}
\sphinxstylestrong{1.} UU Nomor 11 Tahun 2008 tentang Informasi dan Transaksi Elektronik

\begin{sphinxVerbatim}[commandchars=\\\{\},numbers=left,firstnumber=1,stepnumber=1]
Penjelasan
   Pasal 11
      Ayat (1)
         Undang\PYGZhy{}Undang ini memberikan pengakuan secara tegas bahwa meskipun hanya merupakan
         suatu kode, Tanda Tangan Elektronik memiliki kedudukan yang sama dengan tanda
         tangan manual pada umumnya yang memiliki kekuatan hukum dan akibat hukum.
\end{sphinxVerbatim}

\begin{DUlineblock}{0em}
\item[] \sphinxstylestrong{2.} PP 71 Tahun 2019 tentang Penyelenggaraan Sistem dan Transaksi Elektronik
\item[] \sphinxstylestrong{3.} Peraturan Rektor Nomor 30 Tahun 2019 tentang Perubahan Peraturan Rektor Nomor 057 Tahun 2017 tentang Tata Naskah Dinas
\item[] \sphinxstylestrong{4.} Keputusan Rektor Nomor: 479/SK/R/UI/2020 tentang Aplikasi Naskah Dinas Elektronik dan Tanda Tangan Elektronik Tersetifikasi
\end{DUlineblock}


\chapter{Transisi Penggunaan Andieni}
\label{\detokenize{interface:transisi-penggunaan-andieni}}\label{\detokenize{interface::doc}}
Andieni dapat digunakan pada beberapa skenario transisi berikut :
\begin{enumerate}
\sphinxsetlistlabels{\arabic}{enumi}{enumii}{}{.}%
\item {} 
Agenda surat di unit kerja

\item {} 
Distribusi nota dinas di lingkungan UI

\item {} 
Naskah Dinas Elektronik dengan eSign

\end{enumerate}


\section{1. Agenda Surat}
\label{\detokenize{interface:agenda-surat}}
Pada skenario ini, kasus yang terjadi adalah unit anda menerima surat dari unit kerja lain secara fisik, kemudian anda
(sebagai sekretaris) melakukan proses alih media (scanning) dan input nota dinas tersebut di andieni.

\noindent\sphinxincludegraphics{{skenario_1b}.png}


\section{2. Distribusi ND}
\label{\detokenize{interface:distribusi-nd}}
Unit kerja di lingkungan UI, menciptakan nota dinas dengan tanda tangan basah (konvensional) kemudian untuk mempercepat
distribusi naskah maka dikirimkan melalui media elektronik file digital dari nota dinas tersebut. Media elektronik
resmi dalam distribusi naskah dinas di lingkungan UI adalah andieni sebagaimana tercantum dalam Keputusan Rektor Nomor
2635/SK/R/UI/2019 tentang Aplikasi Naskah Dinas Elektronik Universitas Indonesia.

Maka unit anda tidak perlu melakukan input data pada agenda surat, karena data tersebut telah diinput oleh unit pencipta
naskah melalui Andieni.

\noindent\sphinxincludegraphics{{skenario_2}.png}


\section{3. Naskah Dinas Elektronik (eSign)}
\label{\detokenize{interface:naskah-dinas-elektronik-esign}}
Pada kasus kali ini segala aktifitas administrasi dilakukan secara digital, tidak diperlukan lagi tindakan cetak
(baca: print), alih media (scan), dan penandatanganan dokumen dilakukan secara digital.

\noindent\sphinxincludegraphics{{skenario_3}.png}


\chapter{Timeframe Implementasi}
\label{\detokenize{interface:timeframe-implementasi}}
Kantor Arsip sebagai pengampu penerapan naskah dinas elektronik telah merencanakan tahapan penggunaan Naskah Dinas Elektronik
di lingkungan UI sebagai berikut

\noindent\sphinxincludegraphics{{timeframe_2020}.png}


\chapter{Pendaftaran Pengguna Andieni}
\label{\detokenize{daftarPengguna:pendaftaran-pengguna-andieni}}\label{\detokenize{daftarPengguna::doc}}
Andieni diperuntukan bagi pelaksana, sekretaris, dan pimpinan unit kerja. Pada tahap ini pengguna Andieni dibatasi pada
Tenaga Kependidikan dan Struktural di lingkungan UI.


\section{Role Pengguna}
\label{\detokenize{daftarPengguna:role-pengguna}}
Pengguna Andieni dapat dikategorikan menjadi beberapa role sebagai berikut


\begin{savenotes}\sphinxattablestart
\centering
\begin{tabular}[t]{|*{3}{\X{1}{3}|}}
\hline
\sphinxstyletheadfamily 
Role
&\sphinxstyletheadfamily 
Jabatan
&\sphinxstyletheadfamily 
Kewenangan
\\
\hline
Staf
&
Tenaga Kependidikan di unit kerja
&\begin{itemize}
\item {} 
draft konsep

\item {} 
approver konsep

\item {} 
baca naskah masuk

\item {} 
baca disposisi

\end{itemize}
\\
\hline
Sekretaris
&
Sekretariat di unit kerja
&\begin{itemize}
\item {} 
draft konsep

\item {} 
approver konsep

\item {} 
baca naskah masuk seluruh unit

\end{itemize}
\\
\hline
Mailroom
&
Tata usaha atau sebutan lain
&\begin{itemize}
\item {} 
baca naskah masuk seluruh entitas

\item {} 
entri naskah masuk

\end{itemize}
\\
\hline
Pimpinan
&
Pimpinan unit kerja
&\begin{itemize}
\item {} 
draft konsep

\item {} 
approver konsep

\item {} 
baca naskah masuk

\item {} 
baca disposisi

\item {} 
tanda tangan digital (eSign)

\end{itemize}
\\
\hline
\end{tabular}
\par
\sphinxattableend\end{savenotes}


\section{Pendaftaran Pengguna}
\label{\detokenize{daftarPengguna:pendaftaran-pengguna}}
Pendaftaran pengguna Andieni dapat dilakukan \sphinxhref{https://docs.google.com/forms/d/e/1FAIpQLSeJsqzppPpJYVllUe7LMwINVoqJxklAZ-JK\_1HgQgw-odfw8Q/viewform}{DISINI} .
Beberapa data yang perlu diisi antara lain:
\begin{itemize}
\item {} 
Nama Lengkap

\item {} 
NIP/ NUP

\item {} 
Nomor Telepon aktif

\item {} 
Akun SSO UI

\item {} 
Email aktif

\item {} 
Unit Kerja

\item {} 
Jabatan

\end{itemize}


\section{Aktivasi User Andieni}
\label{\detokenize{daftarPengguna:aktivasi-user-andieni}}
TIm Helpdesk Andieni akan melakukan pendaftaran akun SSO anda agar dapat digunakan pada aplikasi Andieni dalam kurun
waktu maksimal 2 x 24 jam.


\chapter{Registrasi eSign}
\label{\detokenize{daftar_eSign:registrasi-esign}}\label{\detokenize{daftar_eSign::doc}}
\begin{sphinxadmonition}{note}{Catatan:}
Menu pendaftaran hanya ditujukan untuk PIMPINAN.
\end{sphinxadmonition}

\begin{sphinxadmonition}{note}{Catatan:}
Fakultas yang akan menerapkan bagi PIMPINAN UNIT dan DOSEN, Maka Dekan Fakultas/ Direktur Program Vokasi/ Sekolah Pascasarjana perlu membuat rekomendasi penerbitan tanda tangan digital untuk yang bersangkutan.
\end{sphinxadmonition}


\section{Login}
\label{\detokenize{daftar_eSign:login}}
Silahkan login ke andieni.ui.ac.id dengan akun SSO anda

\noindent\sphinxincludegraphics{{login_andieni}.png}

\begin{sphinxadmonition}{warning}{Peringatan:}
pastikan akun anda telah terdaftar pada Andieni, hubungi Helpdesk jika anda kesulitan untuk Login
\end{sphinxadmonition}


\section{Menu Registrasi}
\label{\detokenize{daftar_eSign:menu-registrasi}}
Setelah login anda pilih menu pada sisi kanan atas dibawah nama profil anda.
Kemudian klik \sphinxstylestrong{Registrasi BSRE}

\noindent\sphinxincludegraphics{{regBSRE_menu}.png}

Form Registrasi BSRE akan muncul pada layar anda. Isikan seluruh data tersebut sesuai dengan data diri anda.
Penjelasan isian data sebagai berikut:

\noindent\sphinxincludegraphics{{data_eSign}.png}


\begin{savenotes}\sphinxattablestart
\centering
\begin{tabulary}{\linewidth}[t]{|T|T|}
\hline
\sphinxstyletheadfamily 
Data
&\sphinxstyletheadfamily 
Penjelasan
\\
\hline
NIK
&
Nomor Induk Kependudukan (sesuai KTP)
\\
\hline
Nama
&
Sesuai dengan KTP, tanpa gelar akademik atau keagamaan
\\
\hline
Email
&
Gunakan \sphinxstylestrong{email UI anda}, dengan domain @ui.ac.id
\\
\hline
Telepon
&
Nomor telepon anda yang masih aktif
\\
\hline
Kota
&
Kota lokasi bekerja
\\
\hline
Provinsi
&
Provinsi lokasi bekerja
\\
\hline
NIP
&
Nomor Induk Pegawai
\\
\hline
Jabatan
&
\sphinxstylestrong{50 karakter}, tidak menggunakan koma (,) atau dash (\sphinxhyphen{})
\\
\hline
Unit Kerja
&
\sphinxstylestrong{50 karakter}, tidak menggunakan koma (,) atau dash (\sphinxhyphen{})
\\
\hline
Gambar TTD
&
spesimen/ contoh tanda tangan anda, format .jpeg .jpg size: \textless{} 565 KB
\\
\hline
KTP
&
hasil scan ktp anda, format .jpeg .jpg size: \textless{} 565 KB
\\
\hline
Surat Rekomendasi
&
Surat yang diterbitkan oleh Rektor/ Dekan
\\
\hline
\end{tabulary}
\par
\sphinxattableend\end{savenotes}

\begin{sphinxadmonition}{warning}{Peringatan:}
Email, apabila anda menggunakan email dengan domain selain UI seperti @gmail @yahoo maka pendaftaran anda akan \sphinxstylestrong{ditolak} oleh tim layanan BSRE
\end{sphinxadmonition}

\begin{sphinxadmonition}{note}{Catatan:}
Surat Rekomendasi Rektor telah disediakan oleh Helpdesk Andieni (Kantor Arsip), silahkan hubungi helpdesk kami
\end{sphinxadmonition}

Anda perlu mengisi beberapa data berikut sebagai persyaratan penerbitan tanda tangan digital. Setelah anda berhasil
melakukan registrasi \sphinxstylestrong{(tidak ada error)} ditandai dengan highlight berwarna hijau dibagian atas \sphinxstylestrong{(data berhasil disimpan)}
maka data anda telah terkirim ke BSRE. Mohon untuk melakukan konfirmasi  kepada helpdesk Kantor Arsip untuk
konfirmasi pendaftaran anda.


\section{Email OSD}
\label{\detokenize{daftar_eSign:email-osd}}
Anda akan menerima 3 email dari osd {[}otoritas sertifikat digital{]} :
\begin{enumerate}
\sphinxsetlistlabels{\arabic}{enumi}{enumii}{}{.}%
\item {} 
Email pertana \sphinxhyphen{} \sphinxstylestrong{{[}BSrE{]} Login Information} : berisi informasi akun anda untuk masuk ke portal bsre

\item {} 
Email kedua \sphinxhyphen{} \sphinxstylestrong{{[}eSign{]} Pendaftaran Sertifikat} : berisi link untuk membuat passphrase

\item {} 
Email ketiga \sphinxhyphen{} \sphinxstylestrong{{[}BSrE{]} Certificate Issued} : Email ini menandakan sertifikat anda telah terbit dan dapat digunakan pada Andieni

\end{enumerate}

\noindent\sphinxincludegraphics{{email_BSRE}.png}


\section{PASSPHRASE}
\label{\detokenize{daftar_eSign:passphrase}}
\noindent\sphinxincludegraphics{{passphrase}.png}

Passphrase adalah semacam PIN yang \sphinxstylestrong{hanya boleh diketahui oleh pemilik} tanda tangan digital. Passphrase minimal
terdiri dari 8 karakter. Disarankan berupa frase, hindari password umum seperti tanggal lahir, kota lahir

\begin{sphinxadmonition}{warning}{Peringatan:}
lupa akan passphrase maka sertifikat digital harus di batalkan dan membuat penerbitan baru
\end{sphinxadmonition}

Durasi proses penerbitan tanda tangan digital menyesuaikan dengan jumlah penerbitan tanda tangan digital di bagian
layanan dari BSrE. Apabila anda belum mendapatkan ketiga atau salah satu dari email tersebut silahkan kontak helpdesk
Andieni.

\begin{sphinxadmonition}{note}{Catatan:}
eSign berlaku dengan durasi 2 tahun sejak diterbitkan.
\end{sphinxadmonition}


\section{Tidak termasuk Naskah Dinas Elektronik}
\label{\detokenize{daftar_eSign:tidak-termasuk-naskah-dinas-elektronik}}
\begin{sphinxadmonition}{note}{Catatan:}
Peraturan Rektor Nomor 30 Tahun 2019, Pasal 82D menjelaskan tentang tidak termasuk naskah dinas elektronik
\end{sphinxadmonition}
\begin{enumerate}
\sphinxsetlistlabels{\arabic}{enumi}{enumii}{}{.}%
\item {} 
Naskah Dinas diterbitkan berdasarkan Kitab Undang\sphinxhyphen{}Undang Hukum Pidana

\item {} 
Naskah Dinas berdasarkan kesepakatan ditetapkan dalam bentuk tercetak, tanda tangan basah, dan cap dinas

\item {} 
Hukum Tata Usaha Negara bersifat konkret, individual, dan final menimbulkan akibat hukum bagi seseorang atau badan hukum perdata

\item {} 
Naskah Dinas menimbulkan pembebanan keuangan negara

\item {} 
Naskah Dinas merupakan perbuatan hukum perdata

\end{enumerate}


\section{Materi Sosialisasi Balai Sertifikasi Elektronik}
\label{\detokenize{daftar_eSign:materi-sosialisasi-balai-sertifikasi-elektronik}}



\chapter{Cipta Naskah Dinas Elektronik}
\label{\detokenize{ciptanaskah:cipta-naskah-dinas-elektronik}}\label{\detokenize{ciptanaskah::doc}}
Andieni, membantu anda dalam membuat naskah dinas sesuai dengan format Tata Naskah Dinas Universitas Indonesia. Silahkan
login pada portal \sphinxhref{https://andieni.ui.ac.id}{Andieni}

\noindent\sphinxincludegraphics{{login_andieni}.png}

\begin{sphinxadmonition}{note}{Catatan:}
Pastikan bahwa anda telah mendaftarkan akun SSO agar dapat menggunakan Andieni
\end{sphinxadmonition}

Pilih Menu \sphinxstylestrong{Cipta Naskah}

\noindent\sphinxincludegraphics{{menu_kiri}.png}

Pilih tombol \sphinxstylestrong{Tambah Naskah Dinas Baru}

\noindent\sphinxincludegraphics{{tambah_ciptanaskah}.png}

Saat ini tersedia 2(dua) \sphinxstyleemphasis{template} Naskah Dinas dari 19 Jenis Naskah Dinas di lingkungan UI, yaitu Nota Dinas dan Surat Dinas.
Isikan seluruh data yang diperlukan untuk \sphinxstyleemphasis{generate} naskah dinas anda.

\noindent\sphinxincludegraphics{{ciptanaskah}.png}

Form diatas terdiri dari beberapa field dan tombol. Berikut deskripsi dari masing masing field tersebut.


\begin{savenotes}\sphinxattablestart
\centering
\begin{tabulary}{\linewidth}[t]{|T|T|}
\hline
\sphinxstyletheadfamily 
Surat Dinas / Nota Dinas
&\sphinxstyletheadfamily 
Keterangan
\\
\hline
Nomor Naskah
&
otomatis ter\sphinxhyphen{}generate oleh Andieni
\\
\hline
Jenis Naskah
&
Pilih template Surat Dinas atau
Nota Dinas)
\\
\hline
Perihal
&
Text field yang dapat anda ketik
\\
\hline
Tujuan
&
pastikan tujuan anda telah
menggunakan Andieni
\\
\hline
Sifat
&
Rahasia, Sangat Rahasia, Biasa
\\
\hline
Derajat
&
Segera, Sangat Segera, Biasa
\\
\hline
Tembusan
&
pastikan tujuan anda telah
menggunakan Andieni
\\
\hline
Kode Klasifikasi
&
Cukup pilih indeks kegiatan
kode dibuat oleh Andieni
\\
\hline
Approver
&
Pemberi paraf
\\
\hline
Pejabat Penandatangan
&
Pejabat penandatangan dengan eSign
\\
\hline
Upload Lampiran
&
format .pdf maks. 5MB
\\
\hline
Isi Naskah
&
konten dari naskah dinas anda
\\
\hline
\end{tabulary}
\par
\sphinxattableend\end{savenotes}

\sphinxstylestrong{Nomor Naskah} | Nomor naskah otomatis ter\sphinxhyphen{}generate oleh sistem Andieni dengan notasi khusus.

\noindent\sphinxincludegraphics{{ciptanaskah_nomornaskah}.png}

\sphinxstylestrong{Jenis Naskah} | Dropdown saat ini terdiri dari Surat Dinas dan Nota Dinas.

\noindent\sphinxincludegraphics{{ciptanaskah_jenisnaskah}.png}

\sphinxstylestrong{Perihal} | Textfield yang dapat anda ketikkan perihal dari naskah.

\noindent\sphinxincludegraphics{{ciptanaskah_perihal}.png}

\begin{sphinxadmonition}{warning}{Peringatan:}
hindari penulisan perihal dengan judul umum seperti undangan, rapat, karena ini tidak menggambarkan konten naskah.
\end{sphinxadmonition}

\begin{sphinxadmonition}{note}{Catatan:}
Contoh yang benar penulisan perihal antara lain Rapat RKA Tahun 2020 ; Nara Sumber Seminar Nasional Kewirausahaan ; Sosialisasi Naskah Dinas Elektronik.
\end{sphinxadmonition}

\sphinxstylestrong{Approver} | Dropdown yang dapat anda ketikkan jabatan atau nama approver untuk

\noindent\sphinxincludegraphics{{ciptanaskah_approver_0}.png}

\sphinxstylestrong{Sifat}   | Dropdown yang muncul dan wajib diisi apabila jenis naskah Surat Dinas.

\noindent\sphinxincludegraphics{{ciptanaskah_sifat}.png}

\sphinxstylestrong{Tanggal} | Otomatis ter generate pada hari naskah dibuat.

\sphinxstylestrong{Upload Lampiran} | Field untuk menyisipkan lampiran (jika ada)

\noindent\sphinxincludegraphics{{ciptanaskah_lampiran}.png}

\sphinxstylestrong{Tujuan Naskah}   | Apabila jenis naskah Surat Dinas, field akan berupa textfield yang dapat anda ketikkan secara langsung.

\noindent\sphinxincludegraphics{{ciptanaskah_tujuan_S}.png}

Apabila jenis naskah Naskah Dinas, field akan berupa Dropdown yang dapat anda ketikkan
jabatan atau nama tujuan untuk memberikan sugesti pilihan. Tujuan nas kah dapat dipilih
lebih dari satu.

\noindent\sphinxincludegraphics{{ciptanaskah_tujuan_ND}.png}

\sphinxstylestrong{Tembusan}    | Dropdown yang dapat anda ketikkan jabatan atau nama tembusan untuk memberikan sugesti pilihan.
Tembusan dapat dipilih lebih dari satu.

\noindent\sphinxincludegraphics{{ciptanaskah_tembusan}.png}

\sphinxstylestrong{Kode Klasifikasi}    | Dropdown yang dapat anda ketikkan kode klasifikasi u ntuk memberikan sugesti pilihan.

\noindent\sphinxincludegraphics{{ciptanaskah_klasifikasi}.png}

\begin{sphinxadmonition}{warning}{Peringatan:}
kode klasifikasi mengikuti kegiatan atau proses bisnis yang anda lakukan. Jika menulis klasifikasi dengan kata \sphinxstylestrong{"undangan"} maka tidak akan ditemukan di kode klasifikasi.
\end{sphinxadmonition}

\noindent\sphinxincludegraphics{{ciptanaskah_klasifikasi_2}.png}

\sphinxstylestrong{Pejabat penandatangan}  | Dropdown yang dapat anda ketikkan jabatan atau nama pejabat untuk memberikan sugesti pilihan. Penandatangan hanya dapat dipilih satu.

\noindent\sphinxincludegraphics{{ciptanaskah_signer}.png}

\sphinxstylestrong{Derajat} | Dropdown yang menentukan berapa lama naskah harus ditandatangani.

\noindent\sphinxincludegraphics{{ciptanaskah_derajat}.png}

\sphinxstylestrong{Isi Naskah}  | Field berupa text editor untuk melakukan formatting isi naskah.

\noindent\sphinxincludegraphics{{ciptanaskah_isinaskah}.png}

\begin{sphinxadmonition}{note}{Catatan:}
paragraf baru mohon gunakan shift + enter untuk spasi (1 line single spacing), jika menggunakan enter (maka akan menjadi 2 line spacing)
\end{sphinxadmonition}


\chapter{Approver \sphinxhyphen{} Paraf}
\label{\detokenize{ciptanaskah:approver-paraf}}
Pembubuhan paraf pada sebuah naskah dinas bersifat opsional yang berarti tidak semua naskah dinas membutuhkan paraf
dan wajib dibubuhan paraf sebelum ditandatangani.


\section{1. Tambahkan Approver}
\label{\detokenize{ciptanaskah:tambahkan-approver}}
Andieni memiliki fitur agar naskah dinas yang membutuhkan persetujuan \sphinxstyleemphasis{(approver)} disediakan mekanisme secara sistem
melalui field \sphinxstylestrong{Approve Naskah}. Apabila saat membuat naskah, diisikan nama Pejabat pembubuh paraf maka secara sistem Pejabat
tersebut dapat melihat naskah apa yang perlu di paraf melalui menu \sphinxstylestrong{Approve Naskah}

\noindent\sphinxincludegraphics{{ciptanaskah_approver_10}.png}

Approver dapat diisi lebih dari 1 pengguna secara paralel

\noindent\sphinxincludegraphics{{ciptanaskah_approver_2}.png}

Pada menu Approve Naskah, terdapat dua (2) tab, yaitu Menunggu Persetujuan dan Riwayat Persetujuan. Pada tab Menunggu
Persetujuan muncul list naskah yang akan disetujui oleh user.

\noindent\sphinxincludegraphics{{ciptanaskah_approver_13}.png}

\begin{sphinxadmonition}{warning}{Peringatan:}
Baris naskah akan berlatar belakang warna merah jika naskah surat belum disetujui namun telah melewati batas waktu persetujuan naskah. Batas waktu persetujuan naskah menyesuaikan dari \sphinxstylestrong{derajat}.
\end{sphinxadmonition}

\noindent\sphinxincludegraphics{{ciptanaskah_derajat}.png}

Pada tab Riwayat Persetujuan muncul list naskah yang sudah pernah disetujui oleh user.

\noindent\sphinxincludegraphics{{ciptanaskah_approver_4}.png}


\section{2. Detail Naskah}
\label{\detokenize{ciptanaskah:detail-naskah}}
Halaman Detail Naskah dapat dibuka melalui link Detail pada kolom Aksi. Halaman detail berisi detail naskah dan
daftar naskah yang belum disetujui.

\noindent\sphinxincludegraphics{{ciptanaskah_approver_5}.png}

Pada form persetujuan terdapat 3 tombol yaitu tombol Setuju, Tidak Setuju, dan Koreksi.

\begin{DUlineblock}{0em}
\item[] \sphinxstylestrong{a.} Tombol Setuju digunakan untuk menyetujui naskah. Ketika user menekan tombol Setuju maka halaman detail akan menampilkan keterangan : \sphinxstylestrong{“Anda telah menyetujui surat ini.”}
\item[] \sphinxstylestrong{b.} Tombol Tidak Setuju untuk tidak menyetujui naskah. Pada saat menekan tombol Tidak Setuju, kolom Keterangan harus diisi. Sistem akan menampilkan pesan agar user menuliskan alasan tidak setuju seperti dibawah ini.
\end{DUlineblock}

\noindent\sphinxincludegraphics{{ciptanaskah_approver_6}.png}

Ketika user menekan tombol Tidak Setuju maka halaman detail ak an menampilkan keterangan :
\sphinxstylestrong{“Anda telah tidak menyetujui surat ini.”}

\noindent\sphinxincludegraphics{{ciptanaskah_approver_7}.png}

\begin{sphinxadmonition}{note}{Catatan:}
Naskah yang tidak disetujui akan kembali kepada konseptor naskah.
\end{sphinxadmonition}

\begin{DUlineblock}{0em}
\item[] \sphinxstylestrong{c.} Tombol Koreksi digunakan oleh user untuk melakukan koreksi pada naskah. Perbaikan data dapat dilakukan pada semua role pengguna.
\end{DUlineblock}


\section{3. Tombol Cipta Naskah}
\label{\detokenize{ciptanaskah:tombol-cipta-naskah}}
\begin{DUlineblock}{0em}
\item[] \sphinxstylestrong{a.} Tombol Cek Naskah/Format digunakan untuk menampilkan isi naskah dalam bentuk surat yang lengkap \sphinxstyleemphasis{(preview)}
\item[] \sphinxstylestrong{b.} Simpan draft akan menyimpan konsep surat anda
\item[] \sphinxstylestrong{c.} Tombol Selesai digunakan untuk mengakhiri proses koreksi dan menyimpan seluruh perubahan yang dilakukan.
\item[] \sphinxstylestrong{d.} Tombol Batal digunakan untuk mengakhiri proses koreksi dan tidak menyimpan perubahan yang dilakukan.
\end{DUlineblock}

\noindent\sphinxincludegraphics{{ciptanaskah_approver_8}.png}


\chapter{Tanda Tangan eSign}
\label{\detokenize{ciptanaskah:tanda-tangan-esign}}
\noindent\sphinxincludegraphics{{ciptanaskah_approver_9}.png}

Ketika naskah disetujui oleh semua Approver, status naskah akan berubah menjadi: \sphinxstylestrong{“Menunggu Tanda Tangan”.}
Naskah akan muncul pada menu Tanda Tangan Naskah pada tab Menunggu Tanda Tangan pada akun user
Pejabat Penandatangan terkait.

\noindent\sphinxincludegraphics{{ciptanaskah_approver_11}.png}

\begin{sphinxadmonition}{warning}{Peringatan:}
Ketika salah satu dari Approver tidak menyetujui naskah.status naskah akan berubah menjadi : \sphinxstylestrong{Tidak disetujui}.
\end{sphinxadmonition}

\noindent\sphinxincludegraphics{{ciptanaskah_approver_12}.png}

Pada tab Riwayat Tanda Tangan muncul list naskah yang sudah pernah ditandatangani oleh user.

\noindent\sphinxincludegraphics{{ciptanaskah_approver_14}.png}

Pada form penandatanganan terdapat 3 tombol yaitu tombol Tanda Tangan, Tidak Tanda Tangan, dan Koreks Koreksi.

\begin{DUlineblock}{0em}
\item[] \sphinxstylestrong{a.} Tombol Tanda Tangan digunakan untuk menandatangani naskah. Sebelum menekan tombol Tanda Tangan user harus memasukkan passphare yang digunakan untuk koneksi ke BSRE. Ketika passphrase yang dimasukkan salah, sistem akan menampilkan pesan : \sphinxstylestrong{“Kombinasi NIK dan Passphrase Anda Salah.”}
\end{DUlineblock}

\noindent\sphinxincludegraphics{{ciptanaskah_approver_15}.png}

\begin{DUlineblock}{0em}
\item[] \sphinxstylestrong{b.} Ketika user menekan tombol Tanda Tangan dan passphrase yang dimasukkan sudah sesuai, maka halaman detail akan menampilkan keterangan : \sphinxstylestrong{“Anda telah menandatangani surat ini.”}
\end{DUlineblock}

\noindent\sphinxincludegraphics{{ciptanaskah_approver_16}.png}

\begin{sphinxadmonition}{note}{Catatan:}
Setelah pimpinan menandatangani Nota Dinas Elektroni, maka surat akan langsung terdistribusi ke tujuan naskah. Penerima naskah dapat melihatnya pada kolom Menu \sphinxstylestrong{Halaman Naskah} \sphinxhyphen{}\textgreater{} \sphinxstylestrong{Naskah Masuk Internal}
\end{sphinxadmonition}

\begin{DUlineblock}{0em}
\item[] \sphinxstylestrong{c.} Tombol Tidak Tanda Tangan untuk tidak menan datangani naskah. Pada saat menekan tombol Tidak Tangan, kolom Keterangan harus diisi. Sistem akan menampilkan pesan agar user menuliskan alasan tidak ditandatangani seperti dibawah ini.
\end{DUlineblock}

\noindent\sphinxincludegraphics{{ciptanaskah_approver_17}.png}

Ketika user menekan tombol Tidak Tanda Tangan maka halaman detail akan menampilkan keterangan :
\sphinxstylestrong{“Anda telah tidak menandatangani surat ini.”}

\noindent\sphinxincludegraphics{{ciptanaskah_approver_18}.png}


\section{Notasi \& Nomor Naskah Dinas}
\label{\detokenize{ciptanaskah:notasi-nomor-naskah-dinas}}
Naskah dinas yang diciptakan melalui modul cipta naskah akan ter\sphinxhyphen{}generate secara automatis menjadi bentuk file digital
dalam format .pdf .

\begin{DUlineblock}{0em}
\item[] Nota Dinas Elektronik         : ND.e
\item[] \sphinxstylestrong{ND.e\sphinxhyphen{}{[}nomor urut{]}/UN2.{[}kode jabatan{]}/{[}kode klasifikasi{]}/{[}tahun{]}}
\end{DUlineblock}

\begin{DUlineblock}{0em}
\item[] Surat Dinas Elektronik        : S.e
\item[] \sphinxstylestrong{S.e\sphinxhyphen{}{[}nomor urut{]}/UN2.{[}kode jabatan{]}/{[}kode klasifikasi{]}/{[}tahun{]}}
\end{DUlineblock}

\begin{sphinxadmonition}{note}{Catatan:}
Nomor pada konsep atau draft naskah dinas masih berupa \sphinxstyleemphasis{XX} hingga ditandatangani oleh pimpinan penandatangan.
\end{sphinxadmonition}

\begin{sphinxadmonition}{warning}{Peringatan:}
Tanggal pada surat akan berubah menyesuaikan dengan kapan dokumen tersebut ditandatangani secara elektronik. Bukan pada saat surat tersebut dikonsep.
\end{sphinxadmonition}


\chapter{Penerimaan Naskah Masuk}
\label{\detokenize{agendaSurat:penerimaan-naskah-masuk}}\label{\detokenize{agendaSurat::doc}}
Interface pada andieni terdiri dari beberapa bagian, berikut adalah penjelasan bagian tersebut.


\section{Halaman Naskah}
\label{\detokenize{agendaSurat:halaman-naskah}}
Halaman naskah merupakan halaman default dari Andieni, pada bagian ini interaksi anda akan banyak tertuju dalam pemantauan
naskah masuk maupun disposisi masuk.


\subsection{Naskah Internal}
\label{\detokenize{agendaSurat:naskah-internal}}
Naskah masuk internal merupakan bagian yang digunakan untuk interaksi naskah masuk dari internal UI. Baik diterima dalam bentuk tercetak, elektronik, atau langsung melalui Andieni\sphinxhyphen{}CiptaNaskah.

\noindent\sphinxincludegraphics{{skenario_naskah_masuk_internal}.png}


\subsection{Naskah Eksternal}
\label{\detokenize{agendaSurat:naskah-eksternal}}
Bagian ini digunakan untuk mencatat naskah masuk dari eksternal universitas. Baik diterima dalam bentuk tercetak maupun elektronik.

\noindent\sphinxincludegraphics{{skenario_1a}.png}


\section{Input Naskah Masuk}
\label{\detokenize{agendaSurat:input-naskah-masuk}}
Untuk naskah masuk yang anda terima dalam bentuk tercetak atau elektronik di luar sistem Andieni, maka silahkan tambahkan
pada naskah masuk internal (berasal dari entitas di lingkungan UI) dan naskah masuk eksternal (berasal dari entitas di
luar UI).

Klik \sphinxhyphen{}\textgreater{} \sphinxstylestrong{Tambah Naskah Masuk Internal} atau \sphinxstylestrong{Tambah Naskah Masuk Eksternal}

\noindent\sphinxincludegraphics{{naskahmasuk}.png}

\noindent\sphinxincludegraphics{{naskahmasuk_eksternal}.png}

Isikan metadata sesuai dengan naskah yang anda terima, berikut penjelasan setiap field yang perlu diisi:


\begin{savenotes}\sphinxattablestart
\centering
\begin{tabulary}{\linewidth}[t]{|T|T|}
\hline
\sphinxstyletheadfamily 
Field
&\sphinxstyletheadfamily 
Penjelasan
\\
\hline
\sphinxstylestrong{Sifat Naskah}
&
Pilih sesuai dengan sifat naskah (Biasa/ Rahasia/ Sangat Rahasia)
\\
\hline
\sphinxstylestrong{Jenis Naskah}
&
sudah terisi
\\
\hline
\sphinxstylestrong{Asal Naskah}
&
Freetext, silahkan tuliskan unit pengirim
\\
\hline
\sphinxstylestrong{Nomor Naskah}
&
Tuliskan sesuai dengan nomor surat pada naskah
\\
\hline
\sphinxstylestrong{Tanggal Naskah}
&
Isikan tanggal yang tertera pada naskah (bukan tanggal diterima)
\\
\hline
\sphinxstylestrong{Perihal}
&
Isikan dengan perihal surat
\\
\hline
\sphinxstylestrong{Tujuan Distribusi}
&
Pilih pejabat di lingkungan UI
\\
\hline
\sphinxstylestrong{Klasifikasi}
&
Pilih indeks kegiatan sesuai klasifikasi arsip
\\
\hline
\sphinxstylestrong{Isi singkat}
&
Tuliskan informasi penting misal deadline, agenda rapat, dsb
\\
\hline
\sphinxstylestrong{Referensi}
&
Diisi jika surat yang anda terima terkait dengan naskah dinas lain
\\
\hline
\sphinxstylestrong{Unggah}
&
Pilih file yang telah anda scan atau terima secara elektronik
\\
\hline
\end{tabulary}
\par
\sphinxattableend\end{savenotes}

Kemudian klik \sphinxhyphen{}\textgreater{} \sphinxstylestrong{Kirim}

\noindent\sphinxincludegraphics{{kirim}.png}

Maka akan muncul informasi \sphinxstylestrong{"Berhasil tambah data"}

\noindent\sphinxincludegraphics{{databerhasilditambahkan}.png}

Data yang telah ditambahkan akan muncul pada tab \sphinxstylestrong{Baru}

\begin{sphinxadmonition}{note}{Catatan:}
Apabila naskah telah dibuka penerima maka akan berpindah ke tab \sphinxstylestrong{Sudah Diterima}
\end{sphinxadmonition}

\noindent\sphinxincludegraphics{{naskahditerima}.png}


\chapter{Disposisi Naskah}
\label{\detokenize{disposisi:disposisi-naskah}}\label{\detokenize{disposisi::doc}}
Pada bagian ini dapat dilihat disposisi masuk dari atasan atau rekan kerja. Begitu pula disposisi keluar yang ditujukan kepada staf anda.

\noindent\sphinxincludegraphics{{skenario_disposisi}.png}

Pada bagian ini anda telah menerima naskah masuk dan pilih salah satu naskah yang anda baca dan disposisikan.

\noindent\sphinxincludegraphics{{pilihnaskahmasuk}.png}

Silahkan cek data naskah

\noindent\sphinxincludegraphics{{cekdatanaskah}.png}

jika kurang jelas maka anda dapat mengecek pada file terlampir

\noindent\sphinxincludegraphics{{tombolnaskahmasuk}.png}


\section{Isian Data Disposisi}
\label{\detokenize{disposisi:isian-data-disposisi}}
Isian data disposisi yang perlu diisi sebagai berikut


\begin{savenotes}\sphinxattablestart
\centering
\begin{tabulary}{\linewidth}[t]{|T|T|}
\hline
\sphinxstyletheadfamily 
Field
&\sphinxstyletheadfamily 
Penjelasan
\\
\hline
Pengirim
&
Otomatis oleh sistem
\\
\hline
Sifat Naskah
&
Otomatis oleh sistem, sesuai isian data dari pengirim/ pengentri
\\
\hline
Derajat
&
Pilih Segera/ Sangat Segera/ Biasa
\\
\hline
Pilih Jenis Disposisi
&
Arsipkan
Tindaklanjuti
\\
\hline
Pilih Tujuan Disposisi
&
Pilih staf atau pejabat yang dituju, pastikan mereka sudah terdaftar
dan menggunakan Andieni
\\
\hline
Catatan
&
Tuliskan pesan disposisi untuk manajer/ rekan/ staf anda
\\
\hline
Pencatat
&
Otomatis oleh sistem, nama anda akan tercantum
\\
\hline
Lampiran
&
Otomatis oleh sistem, yaitu surat yang anda disposisikan
\\
\hline
Unggah
&
Pilih file tambahan yang akan jadikan lampiran jika ada
\\
\hline
\end{tabulary}
\par
\sphinxattableend\end{savenotes}

Kemudian klik \sphinxhyphen{}\textgreater{} \sphinxstylestrong{Simpan \& kirim disposisi}

\noindent\sphinxincludegraphics{{kirimdisposisi}.png}


\section{Notifikasi Disposisi}
\label{\detokenize{disposisi:notifikasi-disposisi}}
Manajer/ Rekan kerja/ Staf yang anda disposisikan akan menerima email notifikasi seperti dibawah ini

\noindent\sphinxincludegraphics{{notifdisposisi}.png}

Untuk mengecek posisi dimana disposisi tersebut berada pada andieni, maka anda dapat melihat pada menu \sphinxhyphen{}\textgreater{} \sphinxstylestrong{Halaman Naskah}
kemudian pada bagian \sphinxstylestrong{Naskah Disposisi}

\noindent\sphinxincludegraphics{{disposisimasuk}.png}

\begin{DUlineblock}{0em}
\item[] Tab \sphinxstylestrong{Masuk}     : Disposisi masuk yang ditujukan kepada anda
\item[] Tab \sphinxstylestrong{Keluar}    : Disposisi keluar yang anda tujukan ke orang lain
\end{DUlineblock}


\section{Penyelesaian Disposisi}
\label{\detokenize{disposisi:penyelesaian-disposisi}}
Anda dapat melihat disposisi yang masuk kepada anda di bagian \sphinxstylestrong{Halaman Naskah} \sphinxhyphen{}\textgreater{} \sphinxstylestrong{Naskah Disposisi} \sphinxhyphen{}\textgreater{} Tab \sphinxstylestrong{Masuk}.

\noindent\sphinxincludegraphics{{disposisi}.png}

Pilih disposisi anda baca arahan yang diberikan atau cek file surat. Apabila disposisi yang diberikan sudah anda kerjakan
maka klik \sphinxstylestrong{Selesai}.

\noindent\sphinxincludegraphics{{disposisi_selesai}.png}

Anda dapat melihat \sphinxstyleemphasis{update} status pada bagian naskah yang telah anda disposisikan.

\noindent\sphinxincludegraphics{{disposisi_catatan_selesai}.png}

Anda dapat pula melihat history dari disposisi ini berasal, Pada contoh ini diperlihatkan naskah berasal

\noindent\sphinxincludegraphics{{disposisi_history}.png}


\section{Edit Disposisi}
\label{\detokenize{disposisi:edit-disposisi}}
Anda dapat melakukan edit terhadap disposisi yang telah anda buat. Dengan klik tombol edit pada naskah disposisi anda:

\noindent\sphinxincludegraphics{{edit}.png}

Kemudian anda dapat mengubah beberapa data untuk \sphinxstylestrong{Derajat} dan \sphinxstylestrong{Catatan Disposisi}

\noindent\sphinxincludegraphics{{edit_disposisi_data}.png}

Anda dapat mengubah tujuan disposisi pada bagian \sphinxstylestrong{tab Tujuan Disposisi}

\noindent\sphinxincludegraphics{{edit_disposisi_tujuan}.png}


\chapter{Upload eSign}
\label{\detokenize{upload_eSign:upload-esign}}\label{\detokenize{upload_eSign::doc}}
Jika pada modul cipta naskah, Andieni masih terbatas penggunaan eSign pada 2 naskah dinas yaitu Nota Dinas dan Surat Dinas
maka pada Upload eSign ini lebih fleksibel dan dapat digunakan untuk penandatanganan berkas kedinasan UI sesuai keperluan anda.

\begin{sphinxadmonition}{warning}{Peringatan:}
Perhatikan pengecualian untuk Naskah Dinas Elektronik dan sesuaikan dengan prosedur unit terkait.
\end{sphinxadmonition}

Klik \sphinxhyphen{}\textgreater{} \sphinxstylestrong{Upload eSign} pada bagian menu sebelah kiri

Isikan data yang diperlukan untuk pencatatan naskah


\begin{savenotes}\sphinxattablestart
\centering
\begin{tabulary}{\linewidth}[t]{|T|T|}
\hline
\sphinxstyletheadfamily 
Field
&\sphinxstyletheadfamily 
Penjelasan
\\
\hline
\sphinxstylestrong{Sifat Naskah}
&
Pilih sesuai dengan sifat naskah (Biasa/ Rahasia/ Sangat Rahasia)
\\
\hline
\sphinxstylestrong{Jenis Naskah}
&
sudah terisi
\\
\hline
\sphinxstylestrong{Asal Naskah}
&
Freetext, silahkan tuliskan unit pengirim
\\
\hline
\sphinxstylestrong{Nomor Naskah}
&
Tuliskan sesuai dengan nomor surat pada naskah
\\
\hline
\sphinxstylestrong{Tanggal Naskah}
&
Isikan tanggal yang tertera pada naskah (bukan tanggal diterima)
\\
\hline
\sphinxstylestrong{Perihal}
&
Isikan dengan perihal surat
\\
\hline
\sphinxstylestrong{Klasifikasi}
&
Pilih indeks kegiatan sesuai klasifikasi arsip
\\
\hline
\sphinxstylestrong{Isi singkat}
&
Tuliskan informasi penting misal deadline, agenda rapat, dsb
\\
\hline
\sphinxstylestrong{Signer}
&
Pejabat yang akan menandatangani dokumen anda
\\
\hline
\sphinxstylestrong{Unggah}
&
Pilih file yang telah anda scan atau terima secara elektronik
\\
\hline
\end{tabulary}
\par
\sphinxattableend\end{savenotes}

\begin{sphinxadmonition}{warning}{Peringatan:}
upload berkas hanya dalam format PDF
\end{sphinxadmonition}

Pilih berkas yang anda akan tanda tangani.

\noindent\sphinxincludegraphics{{upload_eSign_2}.png}

Cek berkas naskah yang akan anda tanda tangani, file lampiran terdapat pada bagian bawah berkas yang diunggah. Pastikan
nama anda benar tertera sebagai signer (penandatangan).

\noindent\sphinxincludegraphics{{upload_eSign_3}.png}

Jika anda merasa cukup dan bermaksud untuk menandatangani dokumen tersebut dengan eSign. Maka pilih tombol checklist
berwarna hijau pada bagian kanan atas.

\noindent\sphinxincludegraphics{{checklist_hijau}.png}

Kemudian anda akan diminta untuk mengisi passphrase dari sertifikat eSign anda.

\noindent\sphinxincludegraphics{{passphrase_dispo}.png}

Jika berhasil maka akan muncul highlight berkas berhasil ditandatangai dan disebelah kanan atas halaman anda akan
muncul tombol untuk download berkas.

\noindent\sphinxincludegraphics{{upload_eSign_download}.png}

\begin{sphinxadmonition}{note}{Catatan:}
fitur ini hanya digunakan untuk penandatanganan berkas. dokumen yang telah anda tanda tangani belum terkirim secara otomatis ke tujuan naskah.
\end{sphinxadmonition}


\chapter{Verifikasi Dokumen Digital}
\label{\detokenize{verifikasi:verifikasi-dokumen-digital}}\label{\detokenize{verifikasi::doc}}
Anda dapat melakukan verifikasi keaslian sebuah dokumen digital dengan beberapa cara. Berikut penjelasannya:


\section{1. Adobe Reader}
\label{\detokenize{verifikasi:adobe-reader}}
File digital yang telah anda terima dapat di verifikasi keaslian dokumen dan validitas tanda tangan digital dengan Adobe Reader.

Buka file yang anda dapatkan dengan adobe reader. Apabila anda kesulitan, untuk memastikan anda membuka file
dengan adobe reader lakukan langkah berikut.

\begin{DUlineblock}{0em}
\item[] Klik kanan \sphinxstylestrong{\sphinxhyphen{}\textgreater{}} open with \sphinxstylestrong{\sphinxhyphen{}\textgreater{}} pilih Adobe Acrobat
\end{DUlineblock}

\noindent\sphinxincludegraphics{{verifikasi_adobe}.png}

Pada bagian atas file yang telah dibuka, anda akan otomaris melihat \sphinxstylestrong{signature panel}.

\noindent\sphinxincludegraphics{{verifikasi_adobe_2}.png}

Apabila anda klik bagian pada gambar seperti dibawah sesuai urutan. Maka anda dapat melihat profil penandatangan.

\noindent\sphinxincludegraphics{{verifikasi_adobe_3}.png}

\noindent\sphinxincludegraphics{{verifikasi_adobe_4}.png}

Lebih detail tentang penandatangan maka klik \sphinxstyleemphasis{Details} \sphinxhyphen{}\textgreater{} \sphinxstyleemphasis{Subject}

\noindent\sphinxincludegraphics{{verifikasi_adobe_5}.png}


\section{1.A. Trust Certificate}
\label{\detokenize{verifikasi:a-trust-certificate}}
Pada saat pertama kali menampilkan tanda tangan digital, akan didapatkan tanda checklist pada bagian \sphinxstyleemphasis{Signature Panel}
tidak berwarna hijau, melainkan kuning dengan notifikasi seperti berikut

\noindent\sphinxincludegraphics{{verifikasi_adobe_6}.png}

\begin{sphinxadmonition}{note}{Catatan:}
hal ini terjadi karena komputer anda belum mengenal \sphinxstyleemphasis{certificate} dari pemilik tanda tangan
\end{sphinxadmonition}

Untuk trus dokumen anda pada bagian certificate viewer pilih tab \sphinxstylestrong{trust} dan klik \sphinxstylestrong{Add to Trusted Certificate} kemudian
klik \sphinxstylestrong{OK} dan \sphinxstylestrong{OK}.

\begin{sphinxadmonition}{note}{Catatan:}
setalah add to trust certificate, anda perlu untuk tutup Adobe Acrobat anda dan buka kembali file yang anda tuju. Maka \sphinxstyleemphasis{checklist signature} akan menjadi berwarna hijau.
\end{sphinxadmonition}


\section{1.B. Invalid Signature}
\label{\detokenize{verifikasi:b-invalid-signature}}
Apabila sebuah dokumen telah dilakukan modifikasi, maka signature panel akan menunjukan \sphinxstylestrong{checklist berwarna merah}.
Hal ini yang perlu anda pastikan, artinya dokumen yang anda terima sudah tidak asli (palsu).

\noindent\sphinxincludegraphics{{verifikasi_adobe_7}.png}

Untuk mengecek dokumen asli maka anda dapat klik bagian \sphinxstyleemphasis{click to view this version}

\noindent\sphinxincludegraphics{{verifikasi_adobe_8}.png}

\noindent\sphinxincludegraphics{{verifikasi_adobe_10}.png}

maka dapat dilihat telah terjadi perubahan pada kata \sphinxstylestrong{Tembusan:} \sphinxhyphen{}\textgreater{} \sphinxstylestrong{Tembusa}

\noindent\sphinxincludegraphics{{verifikasi_adobe_11}.png}


\section{2. Very DS \sphinxhyphen{} Android Apps}
\label{\detokenize{verifikasi:very-ds-android-apps}}
Anda dapat melakukan pengecekan dengan aplikasi Android Very DS

\noindent\sphinxincludegraphics{{verifikasi_veryds}.png}

Buka file yang anda terima dari android anda, pilih Very DS sebagai pembuka file.

\noindent\sphinxincludegraphics{{verifikasi_veryds_4}.png}

Jika file yang anda terima telah ditanda tangani secara elektronik dan masih autentik

\noindent\sphinxincludegraphics{{verifikasi_veryds_2}.png}

Jika file yang anda terima telah ditanda tangani secara elektronik namun telah diubah oleh seseorang.

\noindent\sphinxincludegraphics{{verifikasi_veryds_3}.png}

Lalu bagaimana dengan file yang tidak memiliki tanda tangan digital, maka Very DS hanya akan berfungsi sebagai PDF Viewer saja.

\noindent\sphinxincludegraphics{{verifikasi_veryds_5}.png}


\section{3. Portal BSRE (tidak direkomendasikan)}
\label{\detokenize{verifikasi:portal-bsre-tidak-direkomendasikan}}
Laman dapat diakses pada \sphinxurl{https://bsre.bssn.go.id/public/verification}

\noindent\sphinxincludegraphics{{verifikasi_bsre}.png}

Jika file yang diverifikasi memiliki tanda tangan digital yang diterbitkan oleh BSRE

\noindent\sphinxincludegraphics{{verifikasi_bsre_2}.png}

Jika file yang diverifikasi tidak memiliki tanda tangan digital

\noindent\sphinxincludegraphics{{verifikasi_bsre_3}.png}

\begin{sphinxadmonition}{warning}{Peringatan:}
verifikasi dengan portal BSRE tidak disarankan, karena saat ini tidak mendukung untuk pengecekan autentikasi dokumen
\end{sphinxadmonition}


\chapter{Pencarian Naskah}
\label{\detokenize{pencarian:pencarian-naskah}}\label{\detokenize{pencarian::doc}}
Anda dapat melakukan pencarian naskah pada beberapa field berikut. Pada bagian atas halaman naskah dan setiap kolom
pada bagian naskah internal, naskah eksternal, dan disposisi.

\noindent\sphinxincludegraphics{{cari_atas}.png}

\noindent\sphinxincludegraphics{{cari_kolom}.png}

apabila anda tidak menemukan naskah pada tab baru, silahkan klik pada tab sudah diterima pada setiap kolom.

\begin{sphinxadmonition}{note}{Catatan:}
Pencarian saat ini belum menghasilkan seluruh naskah dari kategori internal, eksternal, dan disposisi. Masih parsial per kategori.
\end{sphinxadmonition}

Beberapa field yang dapat anda gunakan untuk menemukan naskah yang anda perlukan yaitu, sort berdasarkan

\begin{DUlineblock}{0em}
\item[] Unit Kerja
\item[] Tanggal
\item[] Perihal
\end{DUlineblock}

Keyword yang anda ketik pada kolom pencarian (search) akan merujuk pada perihal dari naskah.


\chapter{Laporan Andieni}
\label{\detokenize{laporan:laporan-andieni}}\label{\detokenize{laporan::doc}}
Anda dapat membuat laporan naskah dinas yang telah diproses di unit kerja anda dengan menu \sphinxstylestrong{Laporan}.

Pilih jenis naskah yang anda akan buat laporan, Naskah, Disposisi Masuk, Disposisi Keluar.

\noindent\sphinxincludegraphics{{laporan_tipe}.png}

Pilih kurun waktu naskah di proses.

\noindent\sphinxincludegraphics{{laporan_pilih_tanggal}.png}

Kemudian klik tombol \sphinxstylestrong{cari}

\noindent\sphinxincludegraphics{{tombol_cari}.png}

Hasil pencarian dapat anda lihat pada bagian dibawah, seperti berikut.

\noindent\sphinxincludegraphics{{hasil_laporan_preview}.png}

Anda dapat menyimpan file yang telah anda proses dengan Andieni dengan menu berikut:

\noindent\sphinxincludegraphics{{unduh_berkas}.png}

\begin{sphinxadmonition}{note}{Catatan:}
Untuk menjaga apabila terjadi kegagalan sistem pada Andieni, kami menyarankan agar para sekretaris unit melakukan \sphinxstyleemphasis{backup} dengan mengunduh laporan naskah secara berkala setiap seminggu sekali.
\end{sphinxadmonition}


\chapter{Helpdesk Andieni}
\label{\detokenize{index:helpdesk-andieni}}
Apabila anda mengalami kesulitan dalam penggunaan Andieni silahkan kontak kami.

\begin{DUlineblock}{0em}
\item[] Kontak   : Yogie
\item[] Whatsapp : +62 813\sphinxhyphen{}1010\sphinxhyphen{}2553
\item[] Telepon  : 021\sphinxhyphen{}29120934
\item[] Email    : \sphinxhref{mailto:arsip@ui.ac.id}{arsip@ui.ac.id}
\end{DUlineblock}



\renewcommand{\indexname}{Indeks}
\printindex
\end{document}